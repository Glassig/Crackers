Metatext blabla.
*Hur många testade vi på?

\section{Effectiveness}

\subsection{Timescale} %Add standardavvikelse
Table \ref{avg_time} declares the average time in seconds it took for the users to complete each version of the game. The data shows that it takes 38\% more time to complete the speech version when compared to the text version.

\begin{table}[h!]
  \centering
  \begin{tabular}{ccc}
    \toprule
    Speech &   & Text\\
    \midrule
    445.22 &   & 310.22\\
    \bottomrule
  \end{tabular}
  \caption{Average Time to complete each version}\label{avg_time}
\end{table}

\subsection{Commands} %Add standardavvikelse
Table \ref{avg_cmd} declares the average amount of commands used before completing the game. The data shows that it takes almost double (99.67\%) the amount of commands to complete the speech version when compared to the text version. This can also be seen in figure \ref{ideal_cmd}, where the average amount of commands used in each version of the game are compared to the ideal number of commands for each version. The ideal takes in consideration a realistic first playthrough of the game, where the player does not ``magically know'' what items are in each room. The ideal number of commands therefore include commands such as ``look around the room'' and ``search <item>'' to reveal a certain other item needed to complete the game. In figure \ref{ideal_cmd} it is also shown that the ideal number of commands are pretty much the same for both versions of the game, making the difference in average number of commands even more significant.

\begin{table}[h!]
  \centering
  \begin{tabular}{ccc}
    \toprule
    Speech &   & Text\\
    \midrule
    64.33 &   & 31.22\\
    \bottomrule
  \end{tabular}
  \caption{Average Number of Commands in each version}\label{avg_cmd} % OMG LOOK HERE!!!
\end{table}

\begin{figure}[h!]
  \centering
  \includegraphics[width=0.8\textwidth]{images/ideal_cmd.jpg}
  \caption{Average amount of commands per game version compared to the ideal number}\label{ideal_cmd}
\end{figure}

Figure \ref{time_cmd} shows the number of commands used over time. As you would expect, the speech version follows the logic of ``the longer time played the more commands used''. However, this is not the case for the text version, where the time played seems irrelevant to the amount of commands used.

\begin{figure}[h!]
  \centering
  \includegraphics[width=0.8\textwidth]{images/time_cmd.jpg} %width=0.8\textwidth scales the image down to 80 percent of the text-width. Keeps the ratio.
  \caption{Number of commands and amount of time played for each user}\label{time_cmd}
\end{figure}

\section{Ease of Use}
%OBS!!!! MAYBE DESCRIPTION OF IDEAL SHOULD BE DESCRIBED SOMEWHERE ELSE
When the user played the game, the amount of commands used was recorded. A command is the entire sentence the user inputted into the game. To have something to compare to, the game was played by us to record what would be the minimum amount of commands needed to be used given that the player would have a lot of luck, or the ``ideal'' play-session. The rules used was that when entering a room, ``check room'' were inputted to get the description. No item in the room were checked, just taken or used, unless the puzzle involve looking at an item (such as a note). The result can be seen in figure \ref{ideal_cmd} where amount of commands users used is compared to the ideal for each game. In the text-version there is roughly 2 times the amount of commands as in the ideal, while in the speach-version it is about 4 times.

\subsection{English Confidence} \label{sec:eng_con}
Each user had to rate their spoken and written english between 1-5, where 1 is not good and 5 is fluent. These users were divided into groups and the average amount of commands, time and score were calculated for each group. The result for the amount of commands can be seen in figure \ref{eng_cmd}. 
\begin{figure}[ht]
  \centering
  \includegraphics[width=0.8\textwidth]{images/english_cmd.jpg}
  \caption{Average amount of commands used per english level-group}\label{eng_cmd}
\end{figure}
The time is put in seconds and is shown in figure \ref{eng_time}.
\begin{figure}[ht]
  \centering
  \includegraphics[width=0.8\textwidth]{images/english_time.jpg}
  \caption{Average time used per english level-group}\label{eng_time}
\end{figure}
The score from the System Usability Scale in relation to the english level is displayed in figure \ref{eng_score}.
\begin{figure}[ht]
  \centering
  \includegraphics[width=0.8\textwidth]{images/english_score.jpg}
  \caption{The Average score from the usability test per english level-group}\label{eng_score}
\end{figure}

%\subsection{User Former Experience}

\section{Satisfaction (SUS)}
Figure \ref{fig:sus_table} shows the average answer on each question after they have been converted according to formulae \ref{eq:convert} in section \ref{sec:sus}. 
\begin{figure}[ht]
  \centering
  \includegraphics[width=0.8\textwidth]{images/sus.jpg}
  \caption{The average corrected score on each statement for each version}\label{fig:sus_table}
\end{figure}

The total average score as calculated in formulae \ref{eq:sum} for each version can be seen in table \ref{tot_score}.
\begin{table}[h!]
  \centering
  \begin{tabular}{ccc}
    \toprule
    Speech &   & Text\\
    \midrule
    54.17 &   & 78.06\\
    \bottomrule
  \end{tabular}
  \caption{Total average score for each version}\label{tot_score}
\end{table}