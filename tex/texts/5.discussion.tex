\section{Comparison of Input Methods} %Diskution av resultaten
After having tested both the text and speech version of the game and had the test users evaluate the different versions, the results show that text input is more effective than speech input. Even though it may consume more time typing a command than speaking it, the text version took much less time overall and a fewer number of commands to complete. One reason behind this may be that the text version handles more synonyms for the relevant verbs than the speech version does, enabling a greater variation of the commands needed to progress through the game. Another reason may be the difference in how the user input can be registered. When speaking into a microphone there are many variations that can occur, such as different pronunciations, speaking volume, light or dark voice, etc., while typing on a keyboard have no such variations. Due to this the speech version have a higher risk of misunderstanding the user input and may force the user to repeat a command several times before the speech recognizer gets it right. 
%Lägg till text om att det stärker tidigare forskning

As seen in \ref{sec:eng_con}, the level of confidence that the person have in their english can affect how they perform using the speech version. If a person think that they are fluent in english, they play for less time and use fewer commands. Figure \ref{eng_cmd} and \ref{eng_time} support this, with speech level 3 averaging 83 commands in about 534 seconds, level 4 about 60 commands and 440 seconds and then level 5 about 58 commands and 394 seconds. The average time for users of english level 4 did not differ much between versions and for the users with english level 5 the difference is more than level 4, however it is significantly less that in level 3. This shows that speech input could have its place, but an analysis of the user base's english level should be done. If the system would be primarily in native english speaking countries it could work fairly equally to a text version.

When it comes to the System Usability Scale, speech rated on an average of 54 points and text an average of about 78 out of 100. Neither system were considered very usable, but text were more usable than speech. Looking at figure \ref{fig:sus_table} it can be seen that with the exception of statement 10, which is concerning how much is needed to be taught before using the system and is about the same between the input methods, speech is consistently worse than text. Speech is rated about 1-1.5 points less than text on statements 1-7, and statement 8 and 9 is rated about 2 points lower. This shows that on almost all aspects, speech needs improvement. 

\section{Sources of Error} %Felkällor
The implementations of Sphinx4 and Stanford POSTagger might not live up their full capacity. There might be other or additional ways to implement them to make the game work even better. For example Sphinx4 does provide tools to train the recognizer, which might have made the speech recognition more accurate. Editing the implementation of Stanford POSTagger would presumably not affect the results since it is implemented exactly the same for both versions of the game. Editing the implementation of Sphinx4, however, may cause changes in the results, since it is solely used in the speech version.

All test users had Swedish as their main spoken and written language, which might affect the accuracy of the speech version more than of the text version. When typing a command all that matters is grammar and spelling, but when speaking a command you also need the right pronunciation. It should also be noted that the users were asked to rate their own English level, so a personal biased could affect their rating. They might actually be better than they think or they might rate themselves higher than they should. If the users were to take some form of test to get a more accurate ranking of their English level it would make the results more reliable.

The results are based on tests performed with 9 different users, which may not be enough to validate them. It would be appropriate to perform additional tests in order to strengthen the results. It might also be profitable to perform tests with users of different age groups and technical skills, since all test users have been between 20-30 years old and studying at KTH.

\section{Future Research}
If this project were to enhanced in the future, it would be recommended to get a better way of ranking the user's english than having them do it themselves. A more varied user-group would also give a wider perspective: different ages, different technical knowledge and different pre-existing knowledge of similar applications. Most of the users in this experiment had played a text-based game before so they had some idea even before of what they were supposed to do.