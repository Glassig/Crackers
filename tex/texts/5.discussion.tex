\section{Comparison of Input Methods} %Diskution av resultaten
The results show that text input is more effective than speech input. Even though it may consume more time typing a command than speaking it, the text version took less time overall and a fewer number of commands to complete. One reason behind this may be that the text version handles more synonyms for the relevant verbs than the speech version does, enabling a greater variation of the commands needed to progress through the game. Another reason may be the difference in how the user input can be registered. When speaking into a microphone there are many variations that can occur, such as different pronunciations, speaking volume, light or dark voice, etc., while typing on a keyboard have no such variations. Due to this the speech version have a higher risk of misunderstanding the user input and may force the user to repeat a command several times before the speech recognizer gets it right. 
%Lägg till text om att det stärker tidigare forskning

As seen in section \ref{sec:eng_con}, the level of confidence that the person has in their English may affect how they perform using the speech version. If a person think that they are fluent in English, they play for less time and use fewer commands. Figure \ref{eng_cmd} and \ref{eng_time} support this theory. This speaks for that speech as an input method could have uses, but before implementing it into a system an analysis of the English level of the user base should be done. If the system would be primarily used in native English speaking countries it might work equally to a system using text input.

When it comes to the System Usability Scale, the speech version rated an average score of 54.17 and the text version an average score of 78.06 out of 100. According to figure \ref{fig:susadj} the usability of the speech and text versions can be considered ``OK'' and ``good'' respectively. This strengthens the conclusion that text is better than speech, but speech is still good enough to place above ``poor''.

\section{Sources of Error} %Felkällor
The implementations of Sphinx4 and Stanford POSTagger might not live up their full capacity. There might be other or additional ways to implement them to make the game work even better. For example Sphinx4 does provide tools to train the recognizer, which might have made the speech recognition more accurate. Editing the implementation of Stanford POSTagger would presumably not affect the resulting difference between versions since it is implemented exactly the same for both versions of the game. Editing the implementation of Sphinx4, however, may cause changes in the results, since it is solely used in the speech version.

All test users had Swedish as their main spoken and written language, which might affect the accuracy of the speech version more than of the text version. When typing a command all that matters is grammar and spelling, but when speaking a command you also need the right pronunciation. It should also be noted that the users were asked to rate their own English level, so a personal bias could affect their rating. They might actually be better than they think or they might rate themselves higher than they should. If the users were to take some sort of test to get a more accurate ranking of their English level it would make the results more reliable.

The results are based on tests performed with 9 different users, which may not be enough to validate them. It would be appropriate to perform additional tests in order to strengthen the results. It might also be profitable to perform tests with users of different age groups and technical skills.

\section{Future Research}
If this project were to be enhanced in the future, it would be recommended to get a better way of ranking the user's English than having them do it themselves. This could be done by having them take an English test of some kind, taking in consideration the difficulty level of English used in the system. Separate tests for spoken and written English may also be a good idea.

A more varied user group with different ages, different technical knowledge and different pre-existing knowledge of similar applications would give more reliable results.

For future implementations of speech recognition using Sphinx4 it may be worth trying out the trainer for the recognizer that is provided. It may for example be used to adapt the speech recognition to English with Swedish accents.
