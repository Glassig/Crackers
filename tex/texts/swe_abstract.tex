Idag använder en ökande mängd system naturliga språkgränssnitt, vilket gör dem enkla och effektiva att använda. Syftet med denna forskning var att få en ökad förståelse för användbarheten av olika inmatningsmetoder för naturliga språkgränssnitt. Detta gjordes genom att skapa två versioner av ett text-baserat spel med ett naturligt språkgränssnitt, där en version använde tal som inmatningsmetod och andra använde text. Tester utfördes sedan med användare som alla spelade igenom båda versionerna av spelet och sedan utvärderade dem individuellt med hjälp av System Usability Scale, ett system för att mäta graden av användbarhet. Det konstaterades att text fungerade bättre som inmatningsmetod ur alla aspekter. Tal fick dock en hög poäng när användarna kände sig säkra på sin engelska kunnighet, vilket talar för möjligheten att använda tal som en inmatningsmetod för naturliga gränssnitt.