\section{Natural Language Interface}
A Natural Language Interface (NLI) is a way for the user to interact with a system or program in a more natural and intuitive way. NLI have many advantages over other systems, such as that it is flexible, people need little training to use it and it can be allowed to do multiple things at once due to uses of pronouns, quantification and context. In general , NLI’s primary function is that they support and deal with the user’s view of the system and translates it into those actually used by the system \citep{Hend}. A few examples of working consumer NLI are Wolfram Alpha, Siri and Google Search.

\subsection{Natural Language Processing}
Natural Language Processing (NLP) explores how computers can be used to understand and manipulate natural language \citep{Gobi}. The input might be text, speech or other. NLP can be used for translation into another language, to comprehend and represent the content of text, to build/search a database or to maintain a dialogue with a user as part of an interface for database/information retrieval \citep{Allen}. NLP can be seen as a backend to an NLI.

\section{Speech Recognition}
Speech Recognition (SR) systems have been researched and developed as a worldwide activity because of the potential this brings for applications, such as to have voice-interactive management, voice dictation and spoken language translation. Although many successes has been had in the development of practical and useful SR systems, there are still limitations to what can be done. The speech signal is one of the most complex signals that humans try to work with. There is also the fact that human’s vocal systems differ between humans and things can be said in different ways. However, various SR systems have been integrated into consumer-technology today (Siri and Google Now, for example), so we are still making advances with the technology. \citep{SR}

\section{Text-Based Adventure Games}
%FIND A SOURCE
The first version of Zork was written in 1977–1979 and it is one of the earliest interactive fiction computer games. Zork distinguished itself in its genre as an especially rich game, in terms of both the quality of the storytelling and the sophistication of its text parser, which was not limited to simple verb-noun commands (e.g. ``hit troll''), but recognized some prepositions and conjunctions (e.g. ``hit the troll with the Elvish sword''). \citep{Zork}

\subsection{Zork}
%USE A SOURCE
zork zork zork...

\section{Previous Research}