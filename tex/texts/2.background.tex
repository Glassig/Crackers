\section{Natural Language Interface}
A Natural Language Interface (NLI) is a way for the user to interact with a system or program in a more natural and intuitive way. NLI have many advantages over other systems, such as that it is flexible, people need little training to use it and it can be allowed to do multiple things at once due to uses of pronouns, quantification and context. In general , NLI’s primary function is that they support and deal with the user’s view of the system and translates it into those actually used by the system \citep{Hend}. A few examples of working consumer NLI are Wolfram Alpha, Siri and Google Search.

\subsection{Natural Language Processing}
Natural Language Processing (NLP) explores how computers can be used to understand and manipulate natural language \citep{Gobi}. The input might be text, speech or other. NLP can be used for translation into another language, to comprehend and represent the content of text, to build/search a database or to maintain a dialogue with a user as part of an interface for database/information retrieval \citep{Allen}. NLP can be seen as a backend to an NLI.

\section{Speech Recognition}
Speech Recognition (SR) systems have been researched and developed as a worldwide activity because of the potential this brings for applications, such as to have voice-interactive management, voice dictation and spoken language translation. Although many successes has been had in the development of practical and useful SR systems, there are still limitations to what can be done. The speech signal is one of the most complex signals that humans try to work with. There is also the fact that human’s vocal systems differ between humans and things can be said in different ways. However, various SR systems have been integrated into consumer-technology today (Siri and Google Now, for example), so we are still making advances with the technology. \citep{SR}

\section{Text-Based Adventure Games}
Text-based games are a form of interactive fiction, which was the first step away from media where the player is only an observer, such as movies and books, to a media in which the player plays a part of the world. In text-based games the player input commands to change the state of the game. The command's form ranges from verb-noun pairs (such as ``go west'') to complex sentences with multiple commands (``open the door with key and then go west'').\citep[page 54-55]{Sweet}

\subsection{Zork}
One of the earliest text-based games was ``Zork, The Great Underground Empire'', which was released in 1980 to critical aclaim. Byte Magazine said that the game was ``[\ldots] entertaining, eloquent, witty and precisely written''  \citep[page 264]{Byte}. Zork's biggest selling point was its ability to accept free-form instructions. Commands could be put in the same sentence and it would still work, for example ``eat the lunch and drink the water'' which would consume both items while satisfying hunger and thirst. This created a level of freedom for the player while still being able to accept more complex input.\citep{Byte}

\section{Usability} \label{usability}
Usability is defined by the ISO-definition as the extent to which a system or product can be used by users in order to achieve specified goals. The focus is on effectiveness, efficiency and satsifaction in a specific context of use \citep{ISO}. In 1986 John Brooks created a way to test a user interface and its usability where he based it on the first edition of the ISO-definition. The idea is that a user tries out a system after which they answer a specific questionnaire concerning the usability of the system. The user should not think for a long period of time or discuss their opinion with anyone before or while filling out the questionnaire. It is important that the user’s initial thought and own experience is recorded. The questionnaire consists of 10 statements and the user must rank each statement by a scale of 1-5, where 1 is “strongly disagree” and 5 is “strongly agree”. Some examples of the statements are “I thought the system was easy to use” and “I thought there was too much inconsistency in this system”, see appendix \ref{Quest} for all the questions. \citep{Broo}

\section{Previous Research}

\subsection{NLI Technology in Computer Games}
A degree-project in Computer Engineering from a technical university in Spain aimed to see if they could improve the usability of text-based games by using a Natural Language Interface. In it they have three games, one where input is simple and strict commands with no levity on what the user can input. The other two uses different types of NLI, one without lexical consistency (just a verb and a noun is allowed without anything extra needed) while the other requires lexical consistency (i.e., forcing the player to use complete and functioning sentences). They found that the system without lexical consistency had the highest usability closely followed by the one with lexical consistency. The strict commands version were rated very low showing that NLI improves usability of a system. \citep{Memo}

\subsection{Degree Project in Computer Science} %Bättre rubrik?
A degree project in computer science carried out at KTH in 2015 researched whether speech recognition is a useful input method for natural language. This was done by first developing a game based on the text-based game Zork, where the user would enter commands in natural language. The game was then used to conduct a usability test which included two groups of test users, where the first group used keyboard input and the second group used speech recognition with Web Speech API. Data regarding the users' efficiency, behaviour and perception of the system was gathered. The results showed that speech recognition has lower efficiency based on time and number of errors. The users did however percieve speech recognition more positively. The conclusion was drawn that speech recognition can be considered a viable input method in cases where efficiency is not of crucial significance. \citep{qvar}