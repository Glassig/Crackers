A system that has a Natural Language Interface (NLI) makes the user able to interact with the system using natural language, which is a language that has developed in the usual way as a method of communicating between people \citep{NatLan}. The input method may vary where some examples are speech, text and body language. A system like this is handy since the user doesn't have to learn a new language to be able to use the system effectively, such as a programming language or specific button sequences.

Development of natural language as an input method for interacting with systems has been an ongoing process since the late 1940s. Back then the work was focused on machine translation with goals such as translating text or speech from one language to another \citep{Jones}. Today implementations of NLIs are highly encouraged and many frequently used systems include it, such as Google Search, Wolfram Alpha, Siri and GPS systems. Many systems with NLIs are open source, meaning anyone can make use of and contribute to the development of even better NLIs.

The research presented in this paper aims to evaluate and compare two different natural language input methods, namely text and speech. This is done by the use of gamification, where the game is inspired by existing text-based adventure games. The evaluation is based on the ISO-definition of usability.

\section{Problem Statement}
Comparing the two natural language input methods text and speech, which one has the highest usability level according to the ISO-definition of usability?

\section{Scope}
The area of natural languages include various types of communication, such as gestures and touch, but in this project the focus has solely been on text and speech. There also exist different kinds of NLIs for both of these natural languages. For example a text based NLI could use strictly grammatical processing and another could be less strict in this matter. We will however focus on one kind of NLI for both text and speech, since the main goal is to compare the usability of these two input methods.

\section{Purpose}
Natural language is flexible and people need little to no training in how to use it. If it can be successfully translated to a system, it would be possible to create interfaces readily usable by a wider array of people and with a lower entry-level for adaptors. The goal of this project is therefore to gather information about the difference in usability between speech and text based NLIs in order to be able to use the results to build more user friendly NLIs in the future.

%\section{Disposition}

\section{Terminology}

\begin{table}[h!]
  \centering
  \begin{tabular}{ccc}
    \toprule
    Expression & Abbrevation & Deffinition\\
    \midrule
    Natural Language & NLI & A way for the user to\\
    Interface & & interact with a system or\\ 
    & & program by the use of\\
    & & human natural language\\
    \\
    Natural Language & NLP & Derives meaning from natural\\
    Processing & & language input and converts\\
    & & it into something the computer\\
    & & can understand and vice versa.\\
    \\
    Gamification & - & TODO \\
    \\
    System Usability & SUS & System to measure level of\\
    Scale & &  usability\\
    \bottomrule
  \end{tabular}
  \caption{Caption for the table.}\label{termin}
\end{table}