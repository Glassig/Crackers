A system that has a Natural Language Interface (NLI) enables the user to interact with the system using natural language, which is a language that has developed as a method of communicating between people \citep{NatLan}. The input method may vary, where some examples are speech, text and body language. A system like this is practical since the user does not have to learn a new language in order to use the system effectively, such as a programming language or specific button sequences.

Development of natural language as an input method for interacting with systems has been an ongoing process since the late 1940s. Back then the work was focused on machine translation with goals such as translating text or speech from one language to another \citep{Jones}. Today implementations of NLIs are highly encouraged and many frequently used systems include it, such as Google Search and Apple's Siri. Many systems with NLIs are open source, meaning anyone can make use of and contribute to the development of even better NLIs.

The research presented in this paper aims to evaluate and compare two different natural language input methods, namely text and speech. This is done by the use of gamification, where the game is inspired by existing text-based adventure games. The evaluation is based on the ISO-definition of usability, which focuses on the effectiveness, efficiency and satsifaction of a product. \citep{ISO}

\section{Problem Statement}
Comparing the two natural language input methods text and speech, which one has the highest usability level according to the ISO-definition of usability?

\section{Scope}
The area of natural languages include additional types of communication other than text and speech, such as body language and touch, although in this research the focus is solely on text and speech. There are also different kinds of NLIs for both of these natural languages. For example one text-based NLI could use strictly grammatical processing while another could be less strict in this matter. In this research a less grammatically strict NLI is implemented for both text and speech, since the main goal is to compare the usability of these two input methods which makes the grammatical strictness less relevant.

\section{Purpose}
This research has been done in order to gain an increased understanding of the usability of text and speech input when it comes to Natural Language Interfaces. This knowledge may then be applied when constructing similar NLIs in the future or to improve the quality of part-of-speech taggers and speech recognition on the whole.

%\section{Disposition}

\section{Terminology}

\begin{table}[ht]
  \centering
  \begin{tabular}{ccc}
    \toprule
    Expression & Abbrevation & Deffinition\\
    \midrule
    Natural Language & NLI & A way for the user to\\
    Interface & & interact with a system or\\ 
    & & program by the use of\\
    & & human natural language\\
    \\
    Natural Language & NLP & Derives meaning from natural\\
    Processing & & language input and converts\\
    & & it into something the computer\\
    & & can understand and vice versa.\\
    \\
    Gamification & - & TODO \\
    \\
    System Usability & SUS & System to measure level of\\
    Scale & &  usability\\
    \bottomrule
  \end{tabular}
  \caption{Short definitions of relevant expressions}\label{termin}
\end{table}