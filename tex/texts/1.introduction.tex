A system that has a Natural Language Interface (NLI) enables the user to interact with the system using natural language, which is a language that has developed as a method of communication between people \citep{NatLan}. The input methods may vary, where some examples are speech, text and body language. A system like this is practical since the user does not have to learn new interaction techniques, such as a programming language or hotkeys, in order to use the system effectively.

Development of natural language as an input method for interacting with systems has been an ongoing process since the late 1940s. At that time the work was focused on machine translation with goals such as translating text or speech from one language to another \citep{Jones}. Today implementations of NLIs are highly encouraged and many popular systems include it, such as Google Search and Apple's Siri. Many systems with NLIs are open source, meaning anyone can make use of and contribute to the development of even better NLIs.

The research presented in this paper aims to evaluate and compare two different natural language input methods, namely text and speech. This is done by the use of gamification, where the game is inspired by existing text-based adventure games. The evaluation is based on the ISO-definition of usability, which focuses on the effectiveness, efficiency and satisfaction of a product. \citep{ISO} This is because the ISO-defenitions are created by experts representing 161 countries and are globally accepted as standard.

\section{Problem Statement}
Comparing the two natural language input methods text and speech, which one has the highest usability level according to the ISO-definition of usability?

\section{Scope}
The area of natural languages include additional types of communication other than text and speech, such as body language and touch, but in this research the focus is solely on text and speech.

\section{Purpose}
This research has been done in order to increase the understanding of the usability of text and speech as input methods for NLIs. This knowledge is meant to contribute to the future development of NLIs with text or speech as input methods, in matters such as which input method to use and the effects of using either method.

%\section{Disposition}
%\newpage
\section{Terminology}

\begin{table}[ht]
  \centering
  \begin{tabular}{ccc}
    \toprule
    Expression & Abbrevation & Deffinition\\
    \midrule
    Gamification & - & The application of typical\\
    & & elements of game playing\\
    & & to other areas of activity \\
    \\
    Natural Language & NLI & A way for the user to\\
    Interface & & interact with a system or\\ 
    & & program by the use of\\
    & & human natural language\\
    \\
    Natural Language & NLP & Derives meaning from natural\\
    Processing & & language input and converts\\
    & & it into something the computer\\
    & & can understand and vice versa\\
    \\
    System Usability & SUS & System to measure level of\\
    Scale & &  usability\\
    \bottomrule
  \end{tabular}
  \caption{Short definitions of relevant expressions}\label{termin}
\end{table}