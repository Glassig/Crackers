Today an increasing amount of systems make use of Natural Language Interfaces (NLIs), which make them easy and efficient to use. The purpose of this research was to gain an increased understanding of the usability of different input methods for NLIs. This was done by implementing two versions of a text-based game with an NLI, where one version used speech as input method and the other used text. Tests were then performed together with 9 test users that all played through both versions of the game and then evaluated them individually using the System Usability Scale. It was found that text was better as input method on all aspects. Speech however scored high when the users felt confident in their English competence, acknowledging the possibility of using speech as input method for NLIs.

To interact with a system using natural language is done by using a Natural Language Interface (NLI) and means
difficulties for both users and the system. The purpose of this study was to examine whether speech recognition is a useful input method for natural language based on the ISO definition of usability. This was done by developing a game based on the classic text-based game Zork, where the user enters commands in natural language, that was then used to conduct a usability test. The test used between-group design on two groups of users where the first group used keyboard input and the second group used speech recognition with Web Speech API. Both qualitative and quantitative data was gathered during the test regarding the users efficiency, behaviour and perception of the system. The results showed that speech recognition has lower efficiency based on time and number of errors. The users did however percieve speech recognition more positively. Speech recognition can thereby be considered a viable input method in cases where efficiency is not of crucial significance.