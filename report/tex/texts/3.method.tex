\section{The Game}
The game that we develop will be inspired by the previously mentioned ZORK. It will consist of a few rooms and tasks to be performed before reaching a victory scenario. There are two games with different story and environments which requires the user to input different commands in order to win. One of the games uses typing to control the game and the other uses speech. We decided to create two games so that an user who has used one control-scheme could still play the other without having the benefit of knowing what is required to win. The game mechanic is the same between the different games.

\section{Testing} 
We let users test both of the systems in order to compare them. First we explain the basics of a text-adventure: there is a description of the room you are in, you type or say simple commands in order to do things. Sometimes you need to interact with something in order to be able to move on further. Then the user gets to play one of the games and upon completion they will answer a short form that will be expanded upon in the following section: ``System Usability Scale''. We record various data, such as amount of tries, time and if there were anything specific that the user failed on multiple times (such as failing to understand a certain part). This is in case we require the users to say a certain word or phrase at one part but the game has a hard time parsing the command for every user. This is done in an attempt to filter out bad programming and game design on our part. 
After the user has played one version and answered the form the user will play the other version and fill in the form again but with regard to the new control scheme. We will do multiple tests with different users and we will alter which game gets played first.

\section{System Usability Scale}
``[...] the usability of any tool or system has to be viewed in terms of the context in which it is used, and its appropriateness to that context'' (John Brooks, 1996). In 1986 John Brooks created a way to test a user interface and its usability. The idea is that a user tries out the system and after which they answer a form. The user should not think for a long period of time or discuss their opinion with anyone before answering the questions, it should be the user’s first thought and own experience that is recorded. The form consists of 10 statements and the user must rank each statement by a scale of 1-5, where 1 is ``strongly disagree'' and 5 is ``strongly agree''. Some examples of the statements are ``I thought the system was easy to use'' and ``I thought there was too much inconsistency in this system''. To calculate your products score, you take all the odd numbered question’s points minus 1 and all the even numbered question’s points are converted by taking 5 minus the point. So if the user answered 2 on question 1 and 2, the final points for the questions are 1 on question 1 and 3 on question 2. Then you sum up the scores and takes the result times 2.5 in order to get a point on the scale between 0-100. A high point is better and a low score means that the interface needs some severe improvements.